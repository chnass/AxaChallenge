% AXA data science challenge
% Brainstorm solutions

\documentclass[12pt, english, a4paper]{article}

\usepackage[T1]{fontenc}
\usepackage[utf8]{inputenc}
\usepackage[english]{babel}  

\usepackage{fancyhdr}
\usepackage{datetime}
\pagestyle{fancy}
\lhead{}
\lfoot{\today\ \currenttime}
\cfoot{\thepage}
\usepackage[colorlinks = true,
            linkcolor = blue,
            urlcolor  = blue,
            citecolor = blue,
            anchorcolor = blue]{hyperref}
\usepackage{bbding}
\usepackage{booktabs}
\usepackage{multirow}
\usepackage{makecell}
\usepackage{todonotes}


\begin{document}

\title{AXA data science challenge}
\author{Christian Nass}
\maketitle

\section{Aufgabenstellung}
Der Fahrradverleih \href{https://citibikenyc.com/how-it-works/bike-rental-nyc}{CitiBike} vermietet in New York über 12.000 Fahrräder an 750 Verleihstationen. 
Somit ist CitiBike eine echte Alternative zu den herkömmlichen Transportmitteln, wie z.B. U-Bahn oder 
Taxi. CitiBike stellt die durch den Verleih gesammelten \href{https://s3.amazonaws.com/tripdata/index.html}{Daten} der Öffentlichkeit zur Verfügung  
(s. bspw. „2023-citibike-tripdata.zip“).  
\newline

Deine Aufgabe als Data Scientist ist es, CitiBike dabei zu helfen diese Daten wertstiftend zu nutzen, 
beispielsweise indem du für CitiBike Kooperationsmöglichkeiten mit einer Versicherung (und/oder 
umgekehrt) skizzierst. Dazu kannst du zusätzlich die öffentlich zugängigen \href{https://data.cityofnewyork.us/Public-Safety/Motor-Vehicle-Collisions-Crashes/h9gi-nx95/about_data}{Daten} des NYPD zu 
Verkehrsunfällen nutzen. 
\newline

Versioniere die Ergebnisse deiner Analyse bitte mit Git (bspw. auf Github). Stelle uns deine Unterlagen 
und deinen Code bitte spätestens am Abend vor unserem gemeinsamen Termin via E-Mail zur 
Verfügung. 
\newline

Viel Erfolg! 

\newpage

\section{Inspect data}

Notebook: \underline{Inspect-Data.ipynb}

\subsection{CitiBike data}
\begin{itemize}
    \item Period of time: June 2013 till September 2025
    \item Columns:\\
          Content self-explanatory based on name\\
          Names vary over the years\\
        \begin{tabular}{l | c | c}
            \toprule
            Column name & 2013 -- 2019 & 2020 -- 2025\\
            \midrule
            tripduration            & \Checkmark & \\
            starttime               & \Checkmark & \Checkmark \\
            stoptime                & \Checkmark & \Checkmark \\
            start station id        & \Checkmark & \Checkmark \\
            start station name      & \Checkmark & \Checkmark \\
            start station latitude  & \Checkmark & \Checkmark \\
            start station longitude & \Checkmark & \Checkmark \\
            end station id          & \Checkmark & \Checkmark \\
            end station name        & \Checkmark & \Checkmark \\
            end station latitude    & \Checkmark & \Checkmark \\
            end station longitude   & \Checkmark & \Checkmark \\
            bikeid                  & \Checkmark & \\
            usertype                & \Checkmark & \Checkmark \\
            birth year              & \Checkmark & \\
            gender                  & \Checkmark & \\
            ride\_id                 & & \Checkmark \\
            rideable\_type          & & \Checkmark \\
            \bottomrule
        \end{tabular}
    \item Problem: Running out of storage saving all the data on my laptop
    \begin{itemize}
        \item Notebook: \underline{Prepare-CitiBike-Data.ipynb}
        \item Slimmed files by removing duplicated information
        \item Storing station name, latitude and longitude only once in a separate file together with their id
        \item Keep only start and end id in tripdata files
    \end{itemize}
\end{itemize}



\subsection{NYPD Motor Vehicle Collisions - Crashes data}
\begin{itemize}
    \item Period of time: August 2012 till 26th October 2025.\\
        Most reportings from 2016 onwards (new system enroled in 2016).
    \item Columns:\\
        \begin{tabular}{l | l }
            \toprule
            Column name & Description\\
            \midrule
            CRASH DATE & Occurrence date of collision\\
            CRASH TIME & Occurrence time of collision\\ 
            BOROUGH & Borough where collision occurred\\ 
            ZIP CODE & Postal code of incident occurrence\\ 
            LATITUDE & Latitude coordinate (EPSG 4326)\\ 
            LONGITUDE & Longitude coordinate (EPSG 4326)\\
            LOCATION & Latitude, Longitude pair\\ 
            ON STREET NAME & Street on which the collision occurred\\ 
            CROSS STREET NAME & Nearest cross street to the collision\\
            OFF STREET NAME & Street address if known\\ 
            NUMBER OF PERSONS INJURED & Number of persons injured\\ 
            NUMBER OF PERSONS KILLED & Number of persons killed\\ 
            NUMBER OF PEDESTRIANS INJURED & Number of pedestrians injured\\ 
            NUMBER OF PEDESTRIANS KILLED & Number of pedestrians killed\\ 
            NUMBER OF CYCLIST INJURED & Number of cyclists injured\\ 
            NUMBER OF CYCLIST KILLED & Number of cyclists killed\\ 
            NUMBER OF MOTORIST INJURED & Number of vehicle occupants injured\\ 
            NUMBER OF MOTORIST KILLED & Number of vehicle occupants killed\\
            VEHICLE TYPE CODE & Type of vehicle based on the selected\\
            & vehicle category: e.g. ATV, bicycle, car, \\
            & escooter, motorcycle (Up to 5)\\   
            CONTRIBUTING FACTOR VEHICLE & Factors contributing to the collision\\
            & for designated vehicle (Up to 5)\\ 
            COLLISION\_ID & Unique record code generated by system\\ 
            \bottomrule
        \end{tabular}
    \item 2.2 million collisions recorded
    \item In about 3\% of the collisions a bicycle is involved (67147)\\
        Most of them with a regular (non-electric) bike (83\%)
\end{itemize}

\newpage


\section{Value-adding opportunities of the data}

The focus is set on cooperation of CitiBike with an insurance company. 
Besides that the data can for sure be used to optimise operations at CitiBike.
The data can also be used to evaluate the infrastructure e.g. by analysing the average time to commute between stations and the number of bike crashes. 


\subsection{Cooperation with an insurance company}

\begin{itemize}
    \item CitiBike company
    \begin{itemize}
        \item Liabilities for inadequate maintenance or defective equipment\\
            ---> Business liability insurance, business legal protection insurance
        \item Theft and vandalism
    \end{itemize}
    \item CitiBike customers
    \begin{itemize}
        \item Bike accident insurance
        \item Bike liability insurance
        \item Traffic legal protection insurance
    \end{itemize}
\end{itemize}

It is not possible to provide recommendations for specific insurance plans or their prices based on the available data.
Nor is it possible to calculate precise risk levels for CitiBike users, since the crash dataset does not indicate whether the bikes involved were rented or privately owned.
However, it can be reasonably assumed that riding a CitiBike involves certain risks, and obtaining insurance coverage against them may be advisable.




\subsection{CitiBike operation}

Main goal is to improve the operations / service.
\newline
\begin{tabular}{l | l}
\toprule
CitiBike duties & Data opportunities\\
\midrule
Define tariffs; Pricing & Not enough information\\
\hline
Order bikes & \multirow{4}{*}{\makecell{Predict future bike usage\\in general and for each station}}\\
Hire staff & \\
Redistribute bikes & \\
Open/close stations & \\
\hline
\multirow{2}{*}{Service of bikes} & Identify bikes with low usage\\
                                  & And bikes with short usage and the same start/end station\\
\hline
Sell ads on bike screens & Acquire customers with shops at frequently visited stations\\
\bottomrule
\end{tabular}
\newline
% \begin{itemize}
%     \item Predict future usage capacity utilisation (in general and for each station)\\
%         ---> Order bikes, Hire new staff, etc.\\
%         ---> Redistribute bikes efficiently \\
%         ---> Open/close stations
%     \item Identify bikes that are not used a lot (even though the station is frequently used), or often have a short usage with the same start and end station
%         ---> Bike might be broken and service is needed
%     \item If you know the position of all bikes at one point it is possible to simulate the number of bikes at each station\\
%         ---> Identify stations at which there are often no bikes available. If the net flow (bikes entering, leaving) is zero, not as easy to identify
%     \item Identify routes that are often used
%         ---> Acquire customers to place ads on the bike screens
% \end{itemize}

If the positions of all bikes are known at a given time, it becomes possible to simulate the number of bikes at each station.
This allows for the identification of stations that frequently have no bikes available.
However, this is difficult to achieve with the data currently available.




\newpage

\section{Data analysis}

This section presents the data analysis with respect to the goals listed in the previous section. 

\subsection{Cooperation with an insurance company}

Notebook: \underline{Analyse-crashes.ipynb}\\
Analyse risks for bike riders to be involved in a crash and their causes. 
There is a significant risk of being involved in a crash as a bike driver. 
This is also true for experienced drivers due to inattention etc. of other road users. 
For this reason every driver should have an insurance. 
CitiBike could over in cooperation with an insurance company a special bike insurance, combining the insurances discussed above. 
Similarly, casual members may be asked before every drive to get the same insurance just for that drive. 
\newline
Main takeaway points regarding risks are listed below:
\begin{itemize}
    \item Number of crashes with a bike increase over the years
    \item Many people use the bikes to commute to work and the most crashes are reported around 5pm, when most people drive back home
    \item Crashes mainly in Manhattan at large roads (lots of traffic) --> Most CitiBike rides start there too
    \item Bikes very vulnerable and take severe injuries in crashes where a bike is involved
    \item In 1 out of 200 reported bike crashes the cyclist died
    \item Crashes are caused more than twice as often from other road users than from bikers
    \item There are many different causes for the crashes\\
    ---> Among the most common are driver inattention / distraction, failure to yield Right-of-Way and traffic control disregarded (this is the same for bikers and cars)
    \item For fatal crashes, additional factors such as alcohol involvement, unsafe speed, and poor road conditions appear more often
    \item For new e-bike the risk of a crash is raised due to inexperience
\end{itemize}

Statements about theft and vanalism cannot be made based on the available data. 
Crashes due to inadequate maintenance or defective equipment seem to be low. 
If liabilities do exist, they could be costly (especially in America :D), so it might still be worth CitiBike protecting itself against them.


\subsection{CitiBike operation}










\newpage

\section{Miscellaneous}

\begin{itemize}
    \item Diffusion map of bikes at different timestamps\\
        Presumably arrows are pointing towards Manhattan in the morning and outwards in the evening
    \item Ride insurance price dependent on time, customer history, etc. (ride duration not known at start)
    \item Many accidents probably not recorded
    \begin{itemize}
        \item Presumably relative low damage/harm but insurance still helpful
        \item Accidents without other people involved are not recorded
    \end{itemize}
\end{itemize}

\end{document}
